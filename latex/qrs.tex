\documentclass[a4paper]{article}
\usepackage{tikz}
\usepackage[left=0cm,top=12mm,right=0cm,bottom=12mm,nohead,nofoot]{geometry}
\usepackage{qrcode}

\usetikzlibrary{calc}

\pagestyle{empty}
\setlength{\parindent}{0cm}

\newcommand{\tilewidth}{70}
\newcommand{\tileheight}{67.7}
\newcommand{\qrsize}{4.5}

\newcommand{\badge}[1]{
    \draw[thin,gray] (0, 0) rectangle ++(\tilewidth mm,\tileheight mm);
    \node at ($ (0,0) ! 0.5 ! (\tilewidth mm, \tileheight mm) + (0,5mm) $) { \qrcode[height=\qrsize cm]{#1} };
    \node[anchor=south,font=\Huge\ttfamily,scale=1.5,transform shape] at (\tilewidth mm / 2,2mm) {#1};
}

\begin{document}

\begin{tikzpicture}[remember picture,overlay]
    \foreach[evaluate={-(\j+1) * \tileheight} as \y] \j in {0,1,2,3} {
        \foreach[evaluate={\i * \tilewidth} as \x,evaluate={int(\j*3+\i)} as \badgenumber] \i in {0,1,2} {
            \begin{scope}[xshift=\x mm,yshift=\y mm]
                \badge{\badgenumber}
            \end{scope}
        }
    }
\end{tikzpicture}

% \foreach \i in {1,10,...,19} {
%     \begin{center}
%         \begin{tikzpicture}
%             \foreach[evaluate={\j*\badgewidth} as \dx] \j in {0,1,2} {
%                 \foreach[evaluate={\k*\badgeheight} as \dy,evaluate={int(\i+\j+(2-\k)*3)} as \index] \k in {0,1,2} {
%                     \begin{scope}[xshift=\dx cm,yshift=\dy cm]
%                         \badge{\index}
%                     \end{scope}
%                 }
%             }
%         \end{tikzpicture}
%     \end{center}
% }



\end{document}
